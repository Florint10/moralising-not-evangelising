\chapter{Introduction}

%Note that you may have multiple \texttt{{\textbackslash}include} statements here, e.g.\ one for each subsection.

%General structure of this chapter should read as follows.  This chapter should be used to motivate your study and answer the question ``Why is this important?''. Also, it should define what you set out to achieve (these will be revisited in the conclusions chapter). You should describe your approach to the Aims and Objectives, including an evaluation part.

In the words of game developer and designer Chris Skaggs: ``Every game, whether you like [it] or not, comes with a worldview.'' \parencite[136]{schut_making_2013}. While both analogue and digital game designers may include values inadvertently, they may also design their games to promote specific beliefs \parencite{bogost_persuasive_2007, flanagan_values_2014}. \textcite{schut_making_2013} define one such group of games. \acp{ECG} feature an unambiguously Christian point of view through their content; they achieve this through the inclusion of Bible stories or passages, obvious displays of gospel morals, or an in-game narrative with an evidently Christian outlook (p.\ 137).

Although this definition appears objective, some may disagree on what can be called ``Christian''. After all, different denominations sometimes advocate for conflicting moral instructions. Hence, to avoid making arbitrary distinctions between which values are truly ``Christian'' and which are not, I shall use the term to describe anything conforming to the teachings of \textit{any} Christian denomination. Similarly, conspicuousness can also be argued to be subjective, since Christian content is less likely to be evident to someone who lacks familiarity with it. Therefore, I shall assume that the audience in question at least familiar with the death and resurrection of Jesus Christ, the concept of sin, and God's forgiveness of sins.

%TODO Replace the first sentence: although the quote is relevant, the original developer might have been talking more about cultural undercurrents in games and Shut might have taken it out of context. 
%TODO the concept of sin: there might be different conceptions of sin

%Although, this is an incomplete definition of what one may refer to as a ``Christian Game'', it does single out games presenting Christian\endnote{Here the term ``Christian'' is admittedly used in a liberal manner. I do this to avoid limiting the applicability of this definition by imposing on what is and is not ``Christian''. Later on, I focus specifically on Puritan, Evangelical and Roman Catholic schools of thought due to their relevance to the argument at hand.} ideology in an explicit manner rather than an implicit one (e.g. through allegorical representations of faith).

\section{Motivation} % why is this a non trivial problem

While \acp{ECG} have existed since the early 1980s\footnote{I based this date on the comprehensive list of religious games compiled by \textcite{gonzalez_religious_2022}. Nevertheless, many early \acp{ECG} in the list remain lost media. For a well-documented example, see Red Sea Crossing (1983) \parencite{lucky_chorus_2023, goldfarb_holy_2012}.}, games developer Wisdom Tree was the first to popularise the concept within Christian circles. Most of their early titles were re-implementations of secular games, including but not limited to those by their parent company, Color Dreams \parencite[287]{bogost_persuasive_2007}. In terms of gameplay, Bible trivia sections and bible quotes were the most notable additions. Most mechanics and rules were copied wholesale from their secular counterparts, though some would be modified to better fit the game's narrative. Another significant deviation was their narratives, which adapted Biblical stories or featured Christian protagonists. 

Wisdom Tree's games were also known for their low production quality. \textit{Bible Adventures}, one of their more popular titles, was known for its poor programming and audio. Before speedrunning the ``Baby Moses'' levels during the 2013 \ac{SGDQ} event, speedrunner Brossentia comments that the ``physics are a \textit{little} iffy'', before a member of the audience retorts ``A \textit{little}?''. Bible adventures for the \ac{NES} also featured level music which could be considered to sound repetitive and thin. All of the game's stages, even ones featuring different Bible stories, loop the same fourteen-second track. Furthermore, many titles in the \ac{NES} library would typically use multiple channels on the console's audio chip to play multiple notes at once\footnote{The soundtracks for Castlevania, Megaman, Kirby's Adventure would be good examples.} However, the track in question only makes use of the triangle channel.

%His opinions remain unchanged even when speedrunning the ``David & Goliath'' levels in \ac{SGDQ} 2014. He explains, ``if you've never played Bible Adventures, you know the physics are a little questionable''.  

%Even infamous titles such as Ghostbusters and Action 52 featured tracks using multiple channels.

%TODO add references for game and quote

% Mention how bible adventures only used the triangle channel for sound (https://www.copetti.org/writings/consoles/nes/#audio), and used the rest for sound effects. IT USES THE SAME MUSIC FOR ALL LEVELS

%\textcite[287]{bogost_persuasive_2007} also notes how the games themselves also tend to verge on the absurd. In other words, the games display a lack of attention to how different game elements --- specifically rules, mechanics, graphics, sound, and narrative --- contribute towards the overall experience for players who have come to expect engaging experiences from games, rather than something resembling a serious game.

%TODO Why do I mention only one list here? I should probs fix this

%TODO Remove this paragraph, and include a sentence saying how although the games had a focus on education, player experience was rarely considered: \textcite[287]{bogost_persuasive_2007} also notes how the games themselves also tend to verge on the absurd. In other words, the games display a lack of attention to how different game elements --- specifically rules, mechanics, graphics, sound, and narrative --- contribute towards the overall experience for players who have come to expect engaging experiences from games, rather than something resembling.

%TODO FIX: Otherwise, their gameplay was mostly identical to that of their secular counterparts. (Could flow better with the previous sentence, plus the gameplay does change a lot - one mechanic deviating can change the whole game!)

%TODO CONFIRM ACCURACY AND ADD PAGE NUMBER: \textcite{bogost_persuasive_2007}, also notes how the games themselves also tend to verge on the absurd,

Wisdom Tree seems to have established a track record for subsequent endeavours. Academics, Christians, and players alike have repeatedly critiqued \acp{ECG} for their excessive educational focus, unoriginal gameplay, and inability to engage players of secular games \parencite{bogost_persuasive_2007, schut_making_2013, moon_channel_why_2023, innocentbystander_why_2009}. Furthermore, after reviewing lists of recent religious and Christian game releases \parencite{gonzalez_religious_2022, noauthor_list_2024}, it becomes apparent that many \acp{ECG} published within the past five years (i.e. between 2024 and 2019) still sideline novel and engaging game experiences in favour of religious instruction.%, resulting in an experience which some have described as ``preachy'' \parencite{schut_making_2013} or ``overly-didactic'' \parencite{moon_channel_why_2023}. For example, gaming platform \textit{TruPlay} (2023) includes several games which resemble well-known mobile titles such as \textit{Plants vs. Zombies} (2009), \textit{Candy Crush} (2012) and \textit{Angry Birds} (2010).

%TODO Fix references, they feel redundant.

%TODO Revise what the sources actually say about the games' inability to engage players of secular games - if not there, find new sources.

%TODO read this article and add as a reference for: "who is also a self-identified Christian"

%TODO Use a better source: Looking at a list of Christian game releases \parencite{noauthor_list_2024},

%TODO Add page numbers and timestamps for where specifically these are mentioned: \parencite{bogost_persuasive_2007, schut_making_2013, moon_channel_why_2023}

%TODO Re-order the sentences so the example is at the end. 

%TODO Give more clear examples regarding to the list from wikipedia

% Therefore, to further the development of ECGs is vital to stop and ask what ourselves what is going wrong? - Focus on moralistic tendency

Innovation in the games industry is hard-fought. However, it is startling to think that after over thirty years, most \ac{ECG} developers have stuck to the same approach of representing Christian faith. Meanwhile, within the same timeframe, the gaming industry as a whole has found many creative and engaging ways of embedding values in games \parencite{bogost_persuasive_2007, flanagan_values_2014}. Why don't \ac{ECG} developers take inspiration from such games? Why have \acp{ECG} remained stagnant for so long?

\section{Aims and Objectives} 
\blindtext

\section{Our Approach} 
\blindtext

\section{Document Structure}
\blindtext
