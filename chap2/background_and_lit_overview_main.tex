\chapter{Background \& Literature Overview}

%In this section you need to explain all the theory required to understand your dissertation (i.e.\ the following chapters). But really in this chapter I am going to show you some examples.

\section{What are Christian Games?}

Before I start my analysis into Christian video games, it makes sense to first clarify what constitutes a Christian game. As games scholar Kevin Schut states in an interview, currently there is no consensus on what makes a game Christian \parencite[8:38]{faith_forms_problem_2024}. For example, \textcite{moon_channel_why_2023} 
defines a Christian video game as one which promotes Christian morals, references Christianity explicitly in its theming, in-game narrative, or marketing, and is deemed Christian by both its intended (presumably Christian) audience and its developer. Meanwhile, \textcite{hartgrove_why_2022} has a much simpler definition. He describes \textit{Hypnospace Outlaw} (2019) as a Christian game simply because its creator, Jay Tholen, identifies as Christian --- regardless of its content.

Lacking an established definition, previous researchers of Christian games have had to create definitions for analysis. \textcite{schut_making_2013} derived definitions for three sub-genres of Christian Games based on several interviews he conducted with Christian developers. \acp{ECG} ``feature Bible stories or passages, very clear presentation of the gospel message, and stories that very openly have a Christian worldview'' (ibid, p.~137). \textit{Big Bible Town} (2010), which aims to teach children about Bible stories, is an example of such a game. Meanwhile, \acp{ACG} ``have stories that are not necessarily openly Christian but have Christian themes or messages underneath them'' (ibid, p.~137). Schut also describes Evangelical Christian video games, which aim to evangelise to non-Christian players. Schut describes this as a hypothetical category, since the interviewees did not provide any examples and Schut was unable to think of any.

Although Schut does not include any examples of \acp{ACG}, he does note that many interviewees referenced the narratives in C.S. Lewis's Narnia books. Hence, they were likely not referring to games like those in the \textit{Halo} series. While they do include several references to Christianity and allusions to themes of resurrection \parencite{paulissen_dark_2018}, these games' narratives do not put much emphasis on these aspects. \textit{Alum} (2015) is more likely a suitable example, since its developers describe it as a ``sort of Christian Allegory along the lines of Pilgrims Progress or The Lion, the Witch, and the Wardrobe'' \parencite{crashable_studios_alum_2015}. Schut further confirms this interpretation when he comments that most of the participants' views on Christian Games were related to ideas of a Christian market demographic (ibid.~p.~140).

\acp{ECG} and \acp{ACG} are primarily differentiated based on how explicitly they present Christian content. However using \textit{Alum} as an example of \acp{ACG} illustrates that they can still be quite explicit. Despite of its self-described Christian Allegorical narrative, several players have criticised its lack of subtlety \parencite {arkane_review_2015, baxted_review_2015, virotti_review_2017}, and it is easy to see why. Although the deity which the protagonist interacts with is called the ``Unfeigned Altruist'' rather than ``Jesus Christ'', his dialogue with the protagonist makes the connection obvious. For example, during the fourth chapter, the Unfeigned Altruist mentions that he died for the protagonist's wife. In the same conversation, he also mentions the following: ``The day is coming when I will be there, in flesh and blood... [...] I *will* come, in the fullness of time... I will come to take every rushlight bearer home, to be with me.'' \parencite[10:04]{123pazu_alum_2015}. Perhaps the interviewees might still consider \textit{Alum} to be allegorical. However, Schut's definition does not elaborate on what it means for narratives to be ``openly Christian'', or what makes a display of gospel morals ``clear''. Furthermore it does not describe what makes a moral display or a narrative ``Christian'' in the first place.

\textcite{gonzalez_born-again_2014} studies Evangelical video games through the broader context of ``religious games''. She begins to define this term by first establishing what ``religion'' refers to for the purposes of her research. This proves to be difficult since Evangelical video games do not strictly abide to theological doctrines. As such, when games such as \textit{Timothy and Titus} rename quantities which could be called ``mana'', ``lives'', and ``health'' to ``faith'', ``hope'', and ``love'', the developers do not expect players to believe that the qualities mentioned in 1 Corinthians 13:13 function in the same way as in the game, or to take these aspects of the game seriously. Anyone hoping to defend these theological inaccuracies would most likely have to relent that ``it's just a game''.

Basing her definition on the work by \textcite{chidester_authentic_2005}, Gonzalez locates religion as the negotiations of human ontology in relation non-humans (such as devils, angels or gods) and the exchanges that occur between them. Hence religious video games are digital games which cause these negotiations. This definition does not presume belief or seriousness, since it only requires the existence of such negotiations and exchanges. In other words, one does not need to believe in religion to acknowledge that it exists. Nevertheless, this definition does not reduce non-human agents to analogies. Gonzalez defines religious organisations as persistent fellowships of humans and non-humans negotiating the boundaries of humanity, and limits the scope of her analysis to games associated with these organisations. If this were not the case, she argues that too many situations could be classified as religious.
 
Gonzalez identifies religious games predominantly by searching their content for sacra. She adopts this term from the research on rites of passage by \textcite{turner_forest_1967} and defines them as materials through which negotiations between humans and non-humans take place. Evangelical video games readapt these sacra from Evangelical organisations and combine them with gameplay in ways that encourage them to recall or explore Evangelical discourse related to that sacra. Specifically, discourse about how the sacra in question (e.g. 1 Corinthians 13:13) determines a Christian’s relationship with the Lord.

%%In addition to in-game content, Gonzales also examines paraludic material (such as including game instructions, catalogue descriptions, and website titles) to determine if a game is religious. In the instance of \textit{Hebrew Learning} (), for example, Gonzales considers it religious because its creator describes it as ``Jewish software''.

Not all religious material in games to be sacra according to Gonzalez; when used in fantasy or blasphemy, the materials begin negotiating different borders. In fantasy, non-humans are portrayed as fictional beings rather than agents whose existence shapes our own. So even though \textit{Castlevania} (1986) includes crosses and holy water, neither would be considered sacra since the enemies they are used against are monsters from horror films and literature. Similarly, the reference to John 1:17 in the Halo series would not be considered sacra since the protagonist fights against a fictional alien alliance. Meanwhile, blasphemous games such as \textit{Run, Jesus Run!} (2010) and \textit{Crucify-me! Jesus} (2000) use religious materials to negotiate a border between different human groups, one of which is deemed as idiotic by the other. Consequently, neither fantasy or blasphemous games qualify as religious.

%TODO talk about serious vs blasphemous dichotomy being a bit abmiguous.

In her analysis, Gonzales does not examine \acp{ACG}. However, \textit{Alum} would still qualify as a religious game under her definition. In the games' second chapter, Alum is given a rushlight, an object which allows him to commune with the Unfeigned Altruist. This item also functions as a hint system for the game's point-and-click puzzles; using the rushlight on the environment triggers dialogue from the Unfeigned Altruist, which is occasionally inspired by scripture. For example, using the rushlight on a daylily flower prompts the following reflection: ``The daylily. It does not spin, not does it toil… And yet, it’s dressed more beautifully than the richest kings and queens.''. This remark references part of Jesus's Sermon on the Mount where he speaks about divine providence Luke 12:27, Matthew 6:28–29\footnote{The use of the words ``spin'' and ``toil'' suggest the developers were likely using either the \textcite{noauthor_english_2001} or the \textcite{noauthor_king_1769}.}. In this way, the game adapts sacra from Christian organisations to engage players with discussions about a Christian's relationship with God.

The definitions by Schut and Gonzalez both include the group of games that prompted this study. However, the definition by Gonzalez is less ambiguous in its interpretation of what classifies as ``religious''. Hence, for the purposes of this dissertation ``Christian digital games'' (or ``Christian games'' for short) shall refer to any digital game which prompts negotiations about what it means to be a human with a relationship to God. Often, these games invite players to engage with pre-existing discussions by readapting sacra from Christian religious organisations.  Following Gonzalez\textquotesingle s approach, I will not consider material to be sacra if they are used in a context of fantasy or blasphemy. Having defined what Christian games are, in the following section I will provide an overview of other researchers\textquotesingle \  efforts to study them.
\section{Related Work}
%\textbf{In this section you need to explain (and reference) similar work in literature}.  Make sure to:

\if{false}
  \begin{itemize}
   \item Give a systematic overview of papers with related/similar work
   \item Highlight similarities/differences to your work (perhaps in the form of a table)
  \end{itemize}

  Note that this section may be sectioned based on the different aspects of your dissertation.
\fi

\section{Summary}
\blindtext
