\chapter{Background \& Literature Overview}

%In this section you need to explain all the theory required to understand your dissertation (i.e.\ the following chapters). But really in this chapter I am going to show you some examples.

\section{What are Christian Games?}

Before I start my analysis into Christian video games, I believe it makes sense to first clarify what constitutes a Christian game. Looking at online discourse, there seems to be no agreement on the term's definition. For example, \textcite{moon_channel_why_2023} 
defines a Christian video game as one which promotes Christian morals, references Christianity explicitly in its theming, in-game narrative, or marketing, and is deemed Christian by both its intended (presumably Christian) audience and its developer. Meanwhile, \textcite{hartgrove_why_2022} has a much simpler definition. He describes \textit{Hypnospace Outlaw} (2019) as a Christian game simply because its creator, Jay Tholen, identifies as Christian --- regardless of its content.

Lacking an established definition, previous researchers of the subject have outlined their own definitions for the sake of analysis. After conducting several interviews with Christian developers, \textcite{schut_making_2013} present four definitions. \acp{ECG} ``feature Bible stories or passages, very clear presentation of the gospel message, and stories that very openly have a Christian worldview'' (p.~137). \textit{Big Bible Town} (2010), which aims to teach children about Bible stories, is an example of such a game. Meanwhile, \acp{ACG} ``have stories that are not necessarily openly Christian but have Christian themes or messages underneath them'' (ibid, p.~137). Schut also describes evangelical Christian video games, which aim to evangelise to non-Christian players. Lastly, Christian-friendly games are family-friendly titles which do not include content controversial to Christians such as \textit{Bejeweled} (2000) or \textit{Words with Friends} (2009).

Although Schut does not include any examples of \acp{ACG}, he does note that many interviewees referenced the narratives in C.S. Lewis's Narnia books. Hence, they were likely not referring to games like those in the \textit{Halo} series. While they do include several references to Christianity and allusions to themes of resurrection \parencite{paulissen_dark_2018}, these games' narratives do not put much emphasis on these aspects. \textit{Alum} (2015) is more likely a suitable example, since its developers describe it as a ``sort of Christian Allegory along the lines of Pilgrims Progress or The Lion, the Witch, and the Wardrobe'' \parencite{crashable_studios_alum_2015}. Schut further confirms this interpretation when he comments that most of the participants' views on Christian Games were related to ideas of a Christian demographic (ibid.~p.~140).

While \acp{ECG} and \acp{ACG} are primarily differentiated based on how explicitly they present Christian content, using \textit{Alum} as an example of \acp{ACG} does raise a counterpoint. Despite of its reportedly Allegorical narrative, several players have criticised its lack of subtlety \parencite {arkane_review_2015, baxted_review_2015, virotti_review_2017}. Perhaps one could differentiate \textit{Alum} from \acp{ECG} given that its narrative avoids direct references to Christianity. For example, the deity which the protagonist interacts with is called the ``Unfeigned Altruist'' rather than ``Jesus Christ'', even if this connection is made clear within the end credits. Nevertheless, Schut does not elaborate on what it means for narratives to be ``openly Christian'', or what makes a display of gospel morals ``clear''. Similarly, his use of the term ``Christian'' is also ambiguous, and does not reveal if the participants were referring to particular interpretations of Christianity.

\textcite{gonzalez_born-again_2014} studies Evangelical video games through the broader context of ``religious games''. She begins to define this term by first establishing what ``religion'' refers to for the purposes of her dissertation. This proves to be difficult since Evangelical video games do not strictly abide to theological doctrines. As such, when games such as \textit{Timothy and Titus} rename quantities more commonly called ``mana'', ``lives'', and ``health'' as ``faith'', ``hope'', and ``love'', the developer does not expect players to believe that the qualities mentioned in 1 Corinthians 13:13 function in the same way as in the game or to take these aspects of the game seriously. Anyone hoping to defend these aspects would most likely have to relent that ``it's just a game''. 

Basing her definition on the work by David Chidester, she locates religion as the negotiations of human ontology in relation non-humans (such as devils, angels or gods) and the exchanges that occur between them. Hence religious video games are digital games which hold these negotiations. This definition does not require belief or seriousness, since it only requires the existence of such negotiations and exchanges. In other words, one does not need to believe in religion to acknowledge that it exists. Gonzalez limits her analysis to games associated with religious organisations, which she defines as persistent fellowships of humans and non-humans negotiating the boundaries of humanity. If this were not the case, she argues that the scope would be too broad.
 
Gonzalez identifies religious games predominantly by searching for sacra within the game’s content. She uses this term in reference to Victor Turner’s research on rites of passage and defines them as materials through which negotiations between humans and non-humans take place. The Evangelical video games readapt these sacra from religious communities and combine them with gameplay in ways that encourage them.


\section{Related Work}
%\textbf{In this section you need to explain (and reference) similar work in literature}.  Make sure to:

\if{false}
  \begin{itemize}
   \item Give a systematic overview of papers with related/similar work
   \item Highlight similarities/differences to your work (perhaps in the form of a table)
  \end{itemize}

  Note that this section may be sectioned based on the different aspects of your dissertation.
\fi

\section{Summary}
\blindtext
