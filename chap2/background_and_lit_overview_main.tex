\chapter{Background \& Literature Overview}

%In this section you need to explain all the theory required to understand your dissertation (i.e.\ the following chapters). But really in this chapter I am going to show you some examples.

\section{Related Work}
%\textbf{In this section you need to explain (and reference) similar work in literature}.  Make sure to:

\if{false}
  \begin{itemize}
   \item Give a systematic overview of papers with related/similar work
   \item Highlight similarities/differences to your work (perhaps in the form of a table)
  \end{itemize}

  Note that this section may be sectioned based on the different aspects of your dissertation.
\fi

\section{What is a Christian Game?}

Before I start my analysis into Christian video games, I believe it makes sense to first clarify what constitutes a Christian game. Looking at online discourse, there seems to be no agreement on the term's definition. For example, \textcite{moon_channel_why_2023} 
defines a Christian video game as one which promotes Christian morals, references Christianity explicitly in its theming, in-game narrative, or marketing, and is deemed Christian by both its intended (presumably Christian) audience and its developer. Meanwhile, \textcite{hartgrove_why_2022} has a much simpler definition. He describes \textit{Hypnospace Outlaw} (2019) as a Christian game simply because its creator, Jay Tholen, identifies as Christian --- regardless of its content.

Lacking an established definition, previous researchers of the subject have outlined their own definitions for the sake of analysis. After conducting several interviews with Christian developers, \textcite{schut_making_2013} present four definitions. \acp{ECG} feature an unambiguously Christian point of view through their content; they achieve this through the inclusion of Bible stories or passages, obvious displays of gospel morals, or an in-game narrative with an evidently Christian outlook. \textit{Big Bible Town} (2010), a game designed to teach Biblical stories to children, is an example of such a game. Meanwhile, \acp{ACG} include narratives leaning towards more implicit depictions of Christian themes and morals, in a similar approach to C.S. Lewis's Narnia books (p.\ 137). Although, Schut does not include any examples, we can note that he was likely not referring to games like \textit{Halo}, which only includes references to Christianity. Allum is more likely a suitable example, as its developers describe it as a `` Christian Allegory along the lines of Pilgrims Progress or The Lion, the Witch, and the Wardrobe''.
\section{Summary}
\blindtext
